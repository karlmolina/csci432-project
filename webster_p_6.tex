\documentclass{article}
\usepackage[utf8]{inputenc}

\title{Kyle Webster-P6}

\begin{document}

\maketitle
Video Link: \href{https://www.youtube.com/watch?v=A1J6g1kiLyE&feature=youtu.be}

\section{Contributions}
I have contributed towards making the video with the algorithm descriptions, an aspect of the K-Means algorithm, paper contributions and reviewer,
and the project management with regards to the LaTex code. This project was difficult to divide into three separate parts of equal weights, but we
agreed that since Karl had the most experience with the P5 library as well as the machine learning algorithms, it was most efficient for him to
handle the algorithm aspect. This left Arnold and I the documentation and the video. We both split the workload as evenly as possible utilizing
Google Drive to share the videos between each other.

\section{Strengths and Weaknesses}
The biggest strength in our project was the clear goal that we had defined at the beginning. Since we had decided which two algorithms and what our
plus one element would be early on, we were able to make rapid progress on the project and create something that was difficult to find online.
During our background research, we found that most of the explainations of the DBSCAN algorithm was either Hindu only, a few lightly discussed videos
from other educational lectures, or with hard to understand descriptions. Because of this, we were able to easily understand what was needed
and what the goals were since Arnold has not had machine learning experience before this project.\\
The biggest weakness was for the project was workload distribution. Because of the wide array of interests and skills, it made separating the
tasks evenly with all members uneven at all times. Since we had a member not familiar with machine learning algorithms, let alone K-Means and DBSCAN,
then there are times where some members had the bulk of the work for a section then others had the bulk of the work for a different section.
This weakness would have been negated by implimenting a Trello board to help keep the goals and tasks organized for each group member.

\end{document}
